% Univariate General Linear Model
% _
% Author: Joram Soch, BCCN Berlin
% E-Mail: joram.soch@bccn-berlin.de
% Edited: 01/03/2019, 12:10


\setcounter{equation}{0}
\section{Univariate General Linear Model} \label{sec:GLM}

\subsection{Likelihood function} \label{sec:GLM-LF}

In the univariate general linear model (GLM), a single measured signal ($y$) is modelled as a linear combination ($\beta$) of predictor variables ($X$), where errors ($\varepsilon$) are assumed to be normally distributed around zero and to have a known covariance structure ($V$), but unknown variance factor ($\sigma^2$):

\begin{equation} \label{eq:GLM}
y = X \beta + \varepsilon, \; \varepsilon \sim \mathrm{N}(0, \sigma^2 V) \; .
\end{equation}

In this equation, $y$ is the $n \times 1$ measured signal, $X$ is the $n \times p$ design matrix, $\beta$ is a $p \times 1$ vector of regression coefficients, $\varepsilon$ is an $n \times 1$ vector of errors, $\sigma^2$ is the variance of these errors and $V$ is an $n \times n$ correlation matrix where $n$ is the number of data points and $p$ ist the number of regressors.

The GLM equation (\ref{eq:GLM}) implies the following \textit{likelihood function}

\begin{equation} \label{eq:GLM-LF-class}
p(y|\beta,\sigma^2) = \mathrm{N}(y; X \beta, \sigma^2 V) = \sqrt{\frac{1}{(2 \pi)^n |\sigma^2 V|}} \, \exp\left[ -\frac{1}{2 \sigma^2} (y-X\beta)^T V^{-1} (y-X\beta) \right]
\end{equation}

which, for mathematical convenience, can also be parametrized as

\begin{equation} \label{eq:GLM-LF-Bayes}
p(y|\beta,\tau) = \mathrm{N}(y; X \beta, (\tau P)^{-1}) = \sqrt{\frac{|\tau P|}{(2 \pi)^n}} \, \exp\left[ -\frac{\tau}{2} (y-X\beta)^T P (y-X\beta) \right]
\end{equation}

using the residual precision $\tau = 1/\sigma^2$ and the $n \times n$ precision matrix $P = V^{-1}$.


\subsection{Maximum likelihood} \label{sec:GLM-MLE}

Classical model estimation proceeds by maximizing the \textit{log-likelihood} (LL)

\begin{equation} \label{eq:GLM-LL}
\mathrm{LL}(\beta,\sigma^2) = \log p(y|\beta,\sigma^2) = - \frac{n}{2} \log(2\pi) - \frac{1}{2} \log \left| \sigma^2 V \right| - \frac{1}{2 \sigma^2} (y - X\beta)^T V^{-1} (y - X\beta)
\end{equation}

which gives rise to \textit{maximum-likelihood} (ML) parameter estimates

\vspace{-0.5em}
\begin{equation} \label{eq:GLM-MLE}
\begin{split}
\hat{\beta} &= (X^T V^{-1} X)^{-1} X^T V^{-1} y \\
\hat{\sigma}^2 &= \frac{1}{n} (y - X\hat{\beta})^T V^{-1} (y - X\hat{\beta})
\end{split}
\end{equation}

that can be used to form $t$- and $F$-statistics

\vspace{-0.5em}
\begin{equation} \label{eq:GLM-tF}
\begin{split}
t &= \frac{c^T \hat{\beta}}{\sqrt{\hat{\sigma}^2 c^T \mathrm{cov}(\hat{\beta}) c}} \\
F &= (C^T \hat{\beta})^T (\hat{\sigma}^2 C^T \mathrm{cov}(\hat{\beta}) C)^{-1} (C^T \hat{\beta})
\end{split}
\end{equation}

where $c$ is a $p \times 1$ \textit{contrast vector}, $C$ is a $p \times q$ \textit{contrast matrix} and

\begin{equation} \label{eq:GLM-cov-beta-est}
\mathrm{cov}(\hat{\beta}) = (X^T V^{-1} X)^{-1} \; .
\end{equation}


\subsection{Prior distribution} \label{sec:GLM-NG-prior}

A conjugate prior distribution relative to the likelihood function given by (\ref{eq:GLM-LF-Bayes}) is the \textit{normal-gamma distribution} over regression coefficients $\beta$ and residual precision $\tau$

\begin{equation} \label{eq:GLM-NG-prior}
p(\beta,\tau) = \mathrm{N}(\beta; \mu_0, (\tau \Lambda_0)^{-1}) \cdot \mathrm{Gam}(\tau; a_0, b_0)
\end{equation}

which can be split into a conditional distribution and a marginal distribution

\vspace{-0.5em}
\begin{equation} \label{eq:GLM-NG-prior-pdf}
\begin{split}
p(\beta|\tau) &= \mathrm{N}(\beta; \mu_0, (\tau \Lambda_0)^{-1}) = \sqrt{\frac{|\tau \Lambda_0|}{(2 \pi)^p}} \exp\left[ -\frac{\tau}{2} (\beta-\mu_0)^T \Lambda_0 (\beta-\mu_0) \right] \\
p(\tau) &= \mathrm{Gam}(\tau; a_0, b_0) = \frac{{b_0}^{a_0}}{\Gamma(a_0)} \, \tau^{a_0-1} \exp[-b_0 \tau]
\end{split}
\end{equation}

where $\mu_0$ and $\Lambda_0$ are the prior mean and the prior precision of $\beta$ and $a_0$ and $b_0$ are the prior shape and rate parameters for $\tau$.


\subsection{Joint likelihood} \label{sec:GLM-NG-JL}

Combining the likelihood function (\ref{eq:GLM-LF-Bayes}) with the prior distribution (\ref{eq:GLM-NG-prior-pdf}), the \textit{joint likelihood function} of the general linear model with normal-gamma priors (GLM-NG) becomes

\vspace{-0.5em}
\begin{equation} \label{eq:GLM-NG-JL1}
\begin{split}
p(y,\beta,\tau) = \; & p(y|\beta,\tau) \, p(\beta,\tau) = p(y|\beta,\tau) \, p(\beta|\tau) \, p(\tau) \\
= \; & \sqrt{\frac{|\tau P|}{(2 \pi)^n}} \, \exp\left[ -\frac{\tau}{2} (y-X\beta)^T P (y-X\beta) \right] \cdot \\
& \sqrt{\frac{|\tau \Lambda_0|}{(2 \pi)^p}} \, \exp\left[ -\frac{\tau}{2} (\beta-\mu_0)^T \Lambda_0 (\beta-\mu_0) \right] \cdot \\
& \frac{{b_0}^{a_0}}{\Gamma(a_0)} \, \tau^{a_0-1} \exp[-b_0 \tau] \; .
\end{split}
\end{equation}

Collecting identical variables gives:

\vspace{-0.5em}
\begin{equation} \label{eq:GLM-NG-JL2}
\begin{split}
p(y,\beta,\tau) = \; & \sqrt{\frac{\tau^{n+p}}{(2 \pi)^{n+p}} |P| |\Lambda_0|} \, \frac{{b_0}^{a_0}}{\Gamma(a_0)} \, \tau^{a_0-1} \exp[-b_0 \tau] \cdot \\
& \exp\left[ -\frac{\tau}{2} \left( (y-X\beta)^T P (y-X\beta) + (\beta-\mu_0)^T \Lambda_0 (\beta-\mu_0) \right) \right] \; .
\end{split}
\end{equation}

\pagebreak
Completing the square over $\beta$ gives:

\vspace{-0.5em}
\begin{equation} \label{eq:GLM-NG-JL3}
\begin{split}
p(y,\beta,\tau) = \; & \sqrt{\frac{\tau^{n+p}}{(2 \pi)^{n+p}} |P| |\Lambda_0|} \, \frac{{b_0}^{a_0}}{\Gamma(a_0)} \, \tau^{a_0-1} \exp[-b_0 \tau] \cdot \\
& \exp\left[ -\frac{\tau}{2} \left( (\beta-\mu_n)^T \Lambda_n (\beta-\mu_n) + (y^T P y + \mu_0^T \Lambda_0 \mu_0 - \mu_n^T \Lambda_n \mu_n) \right) \right] \; .
\end{split}
\end{equation}


\subsection{Posterior distribution} \label{sec:GLM-NG-post}

The \textit{posterior distribution} in the GLM-NG can be evaluated using Bayes' theorem:

\begin{equation} \label{eq:GLM-NG-BT}
p(\beta,\tau|y) = \frac{p(y|\beta,\tau) \, p(\beta,\tau)}{p(y)} \; .
\end{equation}

Since $p(y)$ is just a normalization factor, the posterior is proportional to the joint:

\begin{equation} \label{eq:GLM-NG-post}
p(\beta,\tau|y) \propto p(y|\beta,\tau) \, p(\beta,\tau) = p(y,\beta,\tau) \; .
\end{equation}

From the term in (\ref{eq:GLM-NG-JL3}), we can isolate the posterior distribution over $\beta$:

\vspace{-0.5em}
\begin{equation} \label{eq:GLM-NG-post-beta}
\begin{split}
p(\beta|\tau,y) &= \mathrm{N}(\beta; \mu_n, (\tau \Lambda_n)^{-1}) \\
\mu_n &= \Lambda_n^{-1} (X^T P y + \Lambda_0 \mu_0) \\
\Lambda_n &= X^T P X + \Lambda_0 \; .
\end{split}
\end{equation}

From the remaining term, we can isolate the posterior distribution over $\tau$:

\vspace{-0.5em}
\begin{equation} \label{eq:GLM-NG-post-tau}
\begin{split}
p(\tau|y) &= \mathrm{Gam}(\tau; a_n, b_n) \\
a_n &= a_0 + \frac{n}{2} \\
b_n &= b_0 + \frac{1}{2} (y^T P y + \mu_0^T \Lambda_0 \mu_0 - \mu_n^T \Lambda_n \mu_n) \; .
\end{split}
\end{equation}


\subsection{Log model evidence} \label{sec:GLM-NG-LME}

According to the law of marginal probability, the \textit{model evidence} of the GLM-NG is:

\begin{equation} \label{eq:GLM-NG-ME1}
p(y|m) = \iint p(y|\beta,\tau) \, p(\beta|\tau) \, p(\tau) \, \mathrm{d}\beta \, \mathrm{d}\tau \; .
\end{equation}

According to the law of conditional probability, the integrand is equivalent to the joint:

\begin{equation} \label{eq:GLM-NG-ME2}
p(y|m) = \iint p(y,\beta,\tau) \, \mathrm{d}\beta \, \mathrm{d}\tau \; .
\end{equation}

\pagebreak
In (\ref{eq:GLM-NG-JL3}), we have already evaluated this term as

\vspace{-0.5em}
\begin{equation} \label{eq:GLM-NG-LME1}
\begin{split}
p(y,\beta,\tau) = \; & \sqrt{\frac{\tau^n |P|}{(2 \pi)^n}} \, \sqrt{\frac{\tau^p |\Lambda_0|}{(2 \pi)^p}} \, \frac{{b_0}^{a_0}}{\Gamma(a_0)} \, \tau^{a_0-1} \exp[-b_0 \tau] \cdot \\
& \exp\left[ -\frac{\tau}{2} \left( (\beta-\mu_n)^T \Lambda_n (\beta-\mu_n) + (y^T P y + \mu_0^T \Lambda_0 \mu_0 - \mu_n^T \Lambda_n \mu_n) \right) \right] \; .
\end{split}
\end{equation}

Using the posterior distribution over $\beta$, we can rewrite this as

\vspace{-0.5em}
\begin{equation} \label{eq:GLM-NG-LME2}
\begin{split}
p(y,\beta,\tau) = \; & \sqrt{\frac{\tau^n |P|}{(2 \pi)^n}} \, \sqrt{\frac{\tau^p |\Lambda_0|}{(2 \pi)^p}} \, \sqrt{\frac{(2 \pi)^p}{\tau^p |\Lambda_n|}} \, \frac{{b_0}^{a_0}}{\Gamma(a_0)} \, \tau^{a_0-1} \exp[-b_0 \tau] \cdot \\
& \mathrm{N}(\beta; \mu_n, (\tau \Lambda_n)^{-1}) \, \exp\left[ -\frac{\tau}{2} (y^T P y + \mu_0^T \Lambda_0 \mu_0 - \mu_n^T \Lambda_n \mu_n) \right] \; .
\end{split}
\end{equation}

Now, $\beta$ can be integrated out easily:

\vspace{-0.5em}
\begin{equation} \label{eq:GLM-NG-LME3}
\begin{split}
\int p(y,\beta,\tau) \, \mathrm{d}\beta = \; & \sqrt{\frac{\tau^n |P|}{(2 \pi)^n}} \, \sqrt{\frac{|\Lambda_0|}{|\Lambda_n|}} \, \frac{{b_0}^{a_0}}{\Gamma(a_0)} \, \tau^{a_0-1} \exp[-b_0 \tau] \cdot \\
& \exp\left[ -\frac{\tau}{2} (y^T P y + \mu_0^T \Lambda_0 \mu_0 - \mu_n^T \Lambda_n \mu_n) \right] \; .
\end{split}
\end{equation}

Using the posterior distribution over $\tau$, we can rewrite this as

\begin{equation} \label{eq:GLM-NG-LME4}
\int p(y,\beta,\tau) \, \mathrm{d}\beta = \; \sqrt{\frac{|P|}{(2 \pi)^n}} \, \sqrt{\frac{|\Lambda_0|}{|\Lambda_n|}} \, \frac{{b_0}^{a_0}}{\Gamma(a_0)} \, \frac{\Gamma(a_n)}{{b_n}^{a_n}} \, \mathrm{Gam}(\tau; a_n, b_n) \; .
\end{equation}

Finally, $\tau$ can also be integrated out:

\begin{equation} \label{eq:GLM-NG-LME5}
\iint p(y,\beta,\tau) \, \mathrm{d}\beta \, \mathrm{d}\tau = \; \sqrt{\frac{|P|}{(2 \pi)^n}} \, \sqrt{\frac{|\Lambda_0|}{|\Lambda_n|}} \, \frac{\Gamma(a_n)}{\Gamma(a_0)} \, \frac{{b_0}^{a_0}}{{b_n}^{a_n}} = p(y|m) \; .
\end{equation}

Thus, the \textit{log model evidence} of the GLM-NG is given by

\vspace{-0.5em}
\begin{equation} \label{eq:GLM-NG-LME6}
\begin{split}
\log p(y|m) = \frac{1}{2} & \log |P| - \frac{n}{2} \log (2 \pi)  + \frac{1}{2} \log |\Lambda_0| - \frac{1}{2} \log |\Lambda_n| + \\ & \log \Gamma(a_n) - \log \Gamma(a_0) + a_0 \log b_0 - a_n \log b_n \; .
\end{split}
\end{equation}


\pagebreak
\subsection{Cross-validated LME} \label{sec:GLM-NG-cvLME}

For calculation of the \textit{cross-validated log model evidence} (cvLME), the data are splitted into $S$ subsets. In the training phase, all except one subset of the data are analyzed using a non-informative prior $p_\mathrm{ni}(\beta, \tau)$ with the prior parameters

\begin{equation} \label{eq:GLM-NG-prior-ni}
\mu_0 = 0_{p}, \; \Lambda_0 = 0_{pp} \quad \text{and} \quad a_0 = 0, \; b_0 = 0
\end{equation}

to obtain an informative posterior $p(\beta, \tau|\cup_{j \neq i} y_j)$ using equations (\ref{eq:GLM-NG-post-beta}) and (\ref{eq:GLM-NG-post-tau}). In the testing phase, this informative posterior is then applied as a prior distribution to obtain the out-of-sample log model evidence $\log p(y_i|\cup_{j \neq i} y_j)$ via equation (\ref{eq:GLM-NG-LME6}). Summing up over data subsets yields the cvLME according to equation (\ref{eq:cvLME}).

As one can see from equations (\ref{eq:GLM-NG-post-beta}) and (\ref{eq:GLM-NG-post-tau}), the priors in (\ref{eq:GLM-NG-prior-ni}) are non-informative in the sense that only the data remain to influence the posteriors.


\subsection{Special cases} \label{sec:GLM-NG-Spec}

The \textit{univariate Gaussian with unknown variance} (UGuv) is a special case in which

\begin{equation} \label{eq:GLM-NG-UGuv}
X = 1_n,  \quad \beta = \mu \quad \text{and} \quad V = I_n \; .
\end{equation}

Furthermore, \textit{simple linear regression} (SLR) is a special case of the GLM where

\begin{equation} \label{eq:GLM-NG-SLR}
X = [1_n, x],  \quad \beta = [\beta_0, \beta_1]^T \quad \text{and} \quad V = I_n \; .
\end{equation}

The \textit{one-sample t-test}, the \textit{two-sample t-test}, the \textit{paired t-test} and the \textit{omnibus F-test} can all be emulated as comparisons of general linear models with specific design matrices.


\subsection{Implementation} \label{sec:GLM-Imp}

In \textbf{MATLAB}, maximum likelihood estimates and Bayesian posterior distributions can be obtained via the functions \verb|GLM_MLE| and \verb|GLM_Bayes| while log model evidence and cross-validated LME can be calculated using the functions \verb|GLM_LME| and \verb|GLM_cvLME|. Given an $n \times v$ data matrix $Y$, an $n \times p$ design matrix $X$, an $n \times n$ precision matrix $P$ and a number of data subsets $S$, the cvLME for a GLM-NG is calculated as

\begin{equation} \label{eq:GLM-NG-cvLME-MATLAB}
\texttt{cvLME = GLM\_cvLME(Y, X, P, S);}
\end{equation}

In \textbf{Python}, a GLM object has to be initiated via \verb|glm = cvBMS.GLM(Y, X, V)| and maximum likelihood estimates, Bayesian posterior distributions, log model evidence and cross-validated LME are evaluated via \verb|glm.MLE|, \verb|glm.Bayes|, \verb|glm.LME|, and \verb|glm.cvLME|. Given $Y$, $X$, $V$ and $S$ as above, the cvLME for a GLM-NG is calculated as

\begin{equation} \label{eq:GLM-NG-cvLME-Python}
\texttt{cvLME = cvBMS.GLM(Y, X, V).cvLME(S)}
\end{equation}

In all of the above, $V$ and $P$ default to $I_n$ whereas $S$ defaults to 2 when left empty.